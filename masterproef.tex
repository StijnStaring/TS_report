\documentclass[master=mai,masteroption=eg,english]{kulemt}
\setup{% Verwijder de "%" op de volgende lijn bij UTF-8 karakterencodering
  %inputenc=utf8,
  title={Short-term forecasting of individual household electrical consumption},
  author={Ir. Stijn Staring},
  promotor={Prof.\,dr.\,ir.\ Bart De Moor},
  assessor={Prof.\,dr.\,ir.\ Unknown\and Prof.\,dr.\,ir.\ Unknown},
  assistant={Ir.\ Lola Botman}}
% Verwijder de "%" op de volgende lijn als je de kaft wil afdrukken
%\setup{coverpageonly}
% Verwijder de "%" op de volgende lijn als je enkel de eerste pagina's wil
% afdrukken en de rest bv. via Word aanmaken.
%\setup{frontpagesonly}

% Kies de fonts voor de gewone tekst, bv. Latin Modern
\setup{font=lm}

% Hier kun je dan nog andere pakketten laden of eigen definities voorzien

% Tenslotte wordt hyperref gebruikt voor pdf bestanden.
% Dit mag verwijderd worden voor de af te drukken versie.
\usepackage[pdfusetitle,colorlinks,plainpages=false]{hyperref}
\usepackage{bm}
\setlength\parindent{0pt}
\usepackage{graphicx}
\graphicspath{{Pictures/}}
\usepackage{amsfonts}
\usepackage{amsmath}
\usepackage{wrapfig}
\usepackage{physics}
\usepackage{booktabs}
\usepackage[ruled,vlined]{algorithm2e}
\usepackage{amsthm}
\usepackage{caption}
\usepackage{subcaption}
\usepackage{gensymb}
\usepackage{textgreek}
\usepackage{bm}
\usepackage[table]{xcolor}
\DeclareMathOperator*{\argmin}{argmin}
%\usepackage{flafter}

%%%%%%%
% Om wat tekst te genereren wordt hier het lipsum pakket gebruikt.
% Bij een echte masterproef heb je dit natuurlijk nooit nodig!
\IfFileExists{lipsum.sty}%
 {\usepackage{lipsum}\setlipsumdefault{11-13}}%
 {\newcommand{\lipsum}[1][11-13]{\par Hier komt wat tekst: lipsum ##1.\par}}
%%%%%%%

%\includeonly{chap-n}
\begin{document}

\begin{preface}
	I am very pleased to present my master thesis to complete my study in Artificial Intelligence. Conducting this research was an informative process in	which I was able to apply the knowledge and skills that I gained during my studies. Writing my thesis and thus completing my studies would not have been possible
	without the support of my mentor Lola Botman, PhD student at the KU Leuven. Thank you for the interesting meetings and brainstorm sessions we had. Also, I would like to thank my family for their ongoing support during all phases of my studies. They have always been my biggest fans and I could not have done this without the opportunities they have given me. At last, I want to thank everybody that reads this text. Sit back, relax and enjoy.
\end{preface}

\tableofcontents*

\begin{abstract}
  The \texttt{abstract} environment contains a more extensive overview of
  the work. But it should be limited to one page.

\end{abstract}

%\begin{abstract*}
%  In dit \texttt{abstract} environment wordt een al dan niet uitgebreide
%  Nederlandse samenvatting van het werk gegeven.
%  Wanneer de tekst voor een Nederlandstalige master in het Engels wordt
%  geschreven, wordt hier normaal een uitgebreide samenvatting verwacht,
%  bijvoorbeeld een tiental bladzijden. 
%
%\end{abstract*}

% Een lijst van figuren en tabellen is optioneel
%\listoffigures
%\listoftables
% Bij een beperkt aantal figuren en tabellen gebruik je liever het volgende:
\listoffiguresandtables
% De lijst van symbolen is eveneens optioneel.
% Deze lijst moet wel manueel aangemaakt worden, bv. als volgt:
%\chapter{List of Abbreviations and Symbols}
%\section*{Abbreviations}
%\begin{flushleft}
%  \renewcommand{\arraystretch}{1.1}
%  \begin{tabularx}{\textwidth}{@{}p{12mm}X@{}}
%    LoG   & Laplacian-of-Gaussian \\
%    MSE   & Mean Square error \\
%    PSNR  & Peak Signal-to-Noise ratio \\
%  \end{tabularx}
%\end{flushleft}
%\section*{Symbols}
%\begin{flushleft}
%  \renewcommand{\arraystretch}{1.1}
%  \begin{tabularx}{\textwidth}{@{}p{12mm}X@{}}
%    42    & ``The Answer to the Ultimate Question of Life, the Universe,
%            and Everything'' according to \cite{h2g2} \\
%    $c$   & Speed of light \\
%    $E$   & Energy \\
%    $m$   & Mass \\
%    $\pi$ & The number pi \\
%  \end{tabularx}
%\end{flushleft}

% Nu begint de eigenlijke tekst
\mainmatter

\chapter{Introduction}

There is an increasing awareness of the potential benefits of intelligent energy control of the balance between the distributed energy resources and the demand requirements. Intelligent energy control leads to efficient planning, more satisfied consumers and can even help National Electricity Markets in saving considerable operating and maintenance cost according to \cite{NarjesFallah2018}. Having accurate forecasting models is one of the key conditions to in practise realize intelligent energy control. When electricity consumption forecasting is improved, energy producers can build a better trust with its customers by sending reliable bills. Further, can the electricity producer better estimate the energy demand of the whole customer population. The optimization of the energy production planning is possible, which will lead to cheaper electricity production and increased profit margins. More substantiated decisions can be taken with regard to investments and there will be less need of the more flexible but more expensive electricity installations e.g. diesel engines to catch the deficiencies in electricity production.\\

With the invention of the smart meter in 1977 by \textbf{Paraskevakos} it became possible to communicate the electricity consumption with more clarity to the consumer and the electricity producer. Electricity consumption is nearly recorded real time and it allows for two-way communication between the smart meter and the supplier. As explained in \cite{Depuru2011}: ``By introducing smart meters as a new
component of their smart grid system, an avalanche of immensely useful energy usage information
became available to the energy markets.'' From the availability of more detailed datasets flows the explosion of research in the electrical forecasting techniques and the applicability of more complex models. To tackle the 

Also machine learning tools and neural networks

The reason for the explosion in 


Congestion forecasting during network changes





When forecasting is improved on household scale, the customer can be better informed what the bill is going to be at the end of the month/year.
Energy producer can build a better trust with its customer by sending reliable bills. (Providing good service)
Producent can better estimate the energy demand of the whole customer population. This will lead to cheaper electricity production because a better planning is possible where there is less need of the more 
flexible but more expensive electricity installations e.g. diesel engines.



Having reliable forecasts available will play a decisive role in the exploitation of 


 Accurate and reliable forecasting techniques   




\begin{figure}[hb]
	\centering
	\includegraphics[width=1.0\textwidth]{SmartGrid.png}
	\caption{Electricity grid (Source: KU Leuven thesis proposal).}
	\label{fig:power_image}
\end{figure}


- different kinds of loads

- different often used metrics. 

- make the transition to short term and why this is important. 


\section{Importance of topic}
Individual household forecasting is a complex task because of the high amount of uncertainty
and the volatility of the data. To deal with this it was found in literature that often aggregated signals are forecasted instead. If there are papers that discuss electrical household consumption forecasting, they often use a lot of information about the household which will not be scalable in practice due to privacy concerns. This thesis investigates state of the art time series forecasting techniques based on LSTM neural networks that have as goal to forecast the next day of electrical household consumption, given only limited information.\\

When forecasting is improved on household scale, the customer can be better informed what the bill is going to be at the end of the month/year.
Energy producer can build a better trust with its customer by sending reliable bills. (Providing good service)
Producent can better estimate the energy demand of the whole customer population. This will lead to cheaper electricity production because a better planning is possible where there is less need of the more 
flexible but more expensive electricity installations e.g. diesel engines.

- Like in pooling paper --> talk about methods that are used to reduce complexity/ already written in --> now directly forecasting the individual household consumption --> very complex task. 


- see also table PL. 

- working with real-life data

- wie zijn mijn assessoren? 

- local prediction can be important for local changes to the net --> see oneNote goal PL.

- look at goal of thesis --> meeting and PL goal --> what do they want to do exactly

The weather and calendar data is correlated with the energy load profiles

\section{Problem formulation and link with previous studies}
Now going to forecast individual houses, not aggregated signals. 

Things were I should get familiar with: scikit-learn, Tensorflow, Keras, Pandas, Anaconda, Microsoft Azure --> step in improving software skills.

Not just looking at regular data --> individual household data --> not a day will be the same and only have limited information available. These two together makes the forecasting extremely difficult. 


\section{Thesis objective and structure}
The goal of this thesis is to do short-term load forecasting for individual households. A forecast of the electrical load of a household for 24 hours. 

Develop a model for the future prediction of smart meter electric power consumption. The final
objective is to compare several existing methods

%%% Local Variables: 
%%% mode: latex
%%% TeX-master: "thesis"
%%% End: 

\chapter{Basic data analysis}
\label{cha:1}
In this chapter details of the dataset are introduced and a basic analysis is performed. This includes assessing missing data, seasonality,  influence of temperature and household data, comparing weekdays and weekends, applying an ARIMA model for forecasting.

% Look at the comments during the meeting 'writing the WTK thesis'.
% Make sure that write every part sufficient substantiated.

\section{Introduction to dataset}
% how the dataset is made up --> use the information of the competition.
The data that is used in this thesis is made available for the \href{https://ieee-dataport.org/competitions/ieee-cis-technical-challenge-energy-prediction-smart-meter-data}{IEEE-CIS technical challenge on energy prediction from smart data}. It consists out of data from smart meters about the 1/2 hour granulated electricity consumption of $3248$ households located in the United Kingdom in the year $2017$. Each smart meter collected thus a total of $17520$ measurements that are performed by the the leading international energy provider, E.ON UK plc. Not all the $3248$ smart meters consist of full data as can be seen in Figure \ref{fig:amountNaN} in appendix \ref{app:A}. It can be clearly seen that there are $12$ steps in the amount of missing values. This is because the available data ranges from one month (only December) to a full year of data. This acknowledges that customers may have joined at different times during the year. Additionally, missing values are introduced due to errors in sending/receiving from smart meters.\\
Next to the electricity consumption of the different households, also information is available about the average, minimum and maximum temperature of the day on the location of the smart meter. This data is available at a daily resolution. Also, through voluntary surveys, incomplete information is collected about $2143$ smart meters. This concerns e.g. dwelling type, number of occupants, number of bedrooms etc. Table \ref{tab:attributes} displays all the attributes in appendix \ref{app:A}.\\

%After substituting the missing values as discussed in \ref{s:missing_data}, all the available weeks are averaged out over all the $270$ smart meters that contain a full year of measurements. The result is given by Figure \ref{fig:averaged_week}. A difference that with belgian load profiles is the consumption peak after midnight. This is due to the higher use of the electric storage heaters used in the UK. These systems store electric energy when the electric tariff is low e.g. overnight and releases the heat when the tariff is high.
%

\section{Preprocessing}

Following steps discuss the preprocessing done on the consumption time-series containing measurements for the entire year. 

\subsection{Missing data} \label{s:missing_data}
As discussed above the consumption dataset contains additionally to the missing months also missing data due to sending/receiving errors of the smart meters. When this happens the data of the whole day is lost. Two methods to impute the missing values are compared. Method one substitutes the missing values of a time-serie by the mean of all the measurements done by the meter. Method two replaces the missing values by the mean consumption value of the same moment on the next and previous day. If the next or previous day is also missing, the closest known day is used. The resulting signals can be seen in Figure \ref{fig:mv_mean} and Figure \ref{fig:mv_s} in appendix \ref{app:A}.\\

In order to ascertain which method of the two performs the best, a reference dataset is needed in order to compare the estimated with the true values of the missing measurements. From the original dataset which contain $ 3248 $ meters it was found that for $ 181 $ meters the month March was given without missing data. These $ 181 $ complete signals of the month March are used as reference dataset. In order to create the test data in each of the $ 181  $ meter signals $ 7 $ random days of the month March were removed and estimated by the earlier two methods. The normalized mean square errors, $ MSE_{AN} $ and $ MSE_{mean} $ given by $ \sum_{i=1}^{D} e_i^2  $ and normalized by $ MSE_{mean} $ are given in Figure \ref{fig:mv_result}.

\begin{figure}[h!]
	\centering
	\includegraphics[width=0.8\textwidth]{mv_result.png}
	\caption{Resulting month of March after substitution of the missing values by the mean value of the measurements. }
	\label{fig:mv_result}
\end{figure}

From Figure \ref{fig:mv_result} it can be seen that using method $ 2 $ which estimates the missing values by the mean consumption value of the same moment on the next and previous day, outperforms method $ 1 $ which takes the mean of the signal. Therefore, all the missing values in the consumption dataset are estimated using method $ 2 $ with the only exception the first of January and thirty-one December. If one of these two days are missing, the method $ 1 $ is used because of the absence of two neighbouring days. 

\subsection{Removing outliers}
% some meters don't have a lot of missing values, but have very untraditional output. Two cases are looked into
% 1. big deviation from the average meter. 
% 2. a full day of zeros is included. 
% 3. the moving average changes spectaculair --> fundamental change in the energy consumption
% whitch is hard to forecast for. (not consistent with a normal consumption pattern)
% Weird meters that are identified: 2985, 2984,
% Normal meters: 2979, 2982
% idea to check also for outliers on monthly/weekly scale? 
After the missing values are replaced by estimations, the outliers of the electricity consumption signals are identified.
This is done by looking  at the z-scores of the yearly consumptions. A z-score is calculated using equation \ref{eq:z-score} and assumes that the yearly consumptions are normally distributed around the average consumption. Consumptions that have a very low probability to occur are removed by imposing that $ |z-score| < 3 $.

\begin{equation}
	z-score = \frac{x-\mu}{\sigma}
\end{equation}                      

Figure \ref{fig:z-score} gives the obtained z-values. It can be seen that $ 6 $ meters with an unlikely high or low consumption are removed. 

\begin{figure}[h!]
	\centering
	\includegraphics[width=0.8\textwidth]{z-score.png}
	\caption{Z-scores calculated from the yearly consumptions.}
	\label{fig:z-score}
\end{figure}

After the removal of the the outliers still some untraditional meter measurements were identified. For example there were $ 9 $ meters that had multiple days with zero day consumption measurements. Because it is unlikely that a household produces exactly zero kWh on a day all these $ 9 $ meters were removed. The consumption time-serie of one of the meters is displayed in Figure \ref{fig:zero_con} in appendix \ref{app:A}.\\

Also, there has been looked if there were fundamental changes in the electricity consumption of certain meters. This is further discussed in section \ref{s:Removing of fundamental changes in the consumption load}.


\subsection{Normalization of the data}
% normalize as done in ppt
% downside of this normalilzation that outshooters will have influence.
Normalization is necessary because while absolute consumption differs, relative patterns of human behaviour are more similar \cite{Lago2020}. The patterns in the human behaviour is what a forecasting model is trying to predict and normalization contributes by avoiding the disturbance of different magnitudes in which this human pattern may occur. Every individual household time-serie is normalized based on its maximum and minimum value according to equation \ref{eq:norm}. 

\begin{equation}\label{eq:norm}
	normalized value = \frac{x - x_{min}}{x_{max} - x_{min}}
\end{equation}  

As discussed in section \ref{s:Basic analysis} the average is taken over all the normalized time-series to obtain a single signal.\textbf{Ask if this is good??}  Because the maximum is taken into account during the normalization, measurement out shooters have an influence on the normalization. 


\subsection{Removing of fundamental changes in the consumption load}  
% done after normalization.
After normalization of all the individual time-series it is looked for fundamental changes in the consumption load due for example when an extra person lives in the house or when systems are installed that use a lot of electricity, during the year. An example of such a time-serie can be seen in Figure \ref{fig:fundamentel_change} in appendix \ref{app:A}.
These changes are identified by looking at the maximum difference of the minimum and maximum rolling mean consumption over $ 7 $ days for each individual meter. If this difference can not anymore be explained by the dependency on the temperature, it is assumed that a fundamental change in electricity consumption took place. It is desired that the mean consumption doesn't change much during the year. Figure \ref{fig:fund_change} shows all the maximum differences between the minimum and maximum weekly rolling averages. The red line shows the cutoff and the smart meters above this line are removed. In total $ 211 $ smart meters remain to be used in the ``Basic analysis''. In Figure \ref{fig:f_c} in appendix \ref{app:A} the time-serie with the new maximum difference between the minimum and maximum weekly rolling averages is given.

\begin{figure}[h!]
	\centering
	\includegraphics[width=0.8\textwidth]{fund_change.png}
	\caption{The maximum differences between the minimum and maximum weekly rolling averages for all the different time-series.}
	\label{fig:fund_change}
\end{figure}



%\cite{NarjesFallah2018}

\section{Basic analysis}\label{s:Basic analysis}
% Aggregation of the different signals is necessary in order to be able to make predictions.
Finally, the average is taken over all the remaining $211$ time-series to obtain a single signal. An individual household consumption time-serie is too much subdued to complex and personal decisions that cause increases or decreases of the consumption. It is extremely hard to capture all theses effects in a single model. By aggregation of the individual time-series by taking the average, this noisy individual behaviour is mitigated. The aggregated signal is now modelled and the increase or decrease of the consumption can be explained by a small set of variables. The aggregated signal can be seen as a ``virtual distribution substation'' as discussed in \cite{Hoverstad2015}. Typical variables used in a forecasting model are: past electricity consumption loads, weather information, calendar information and error-correction terms \cite{loadforecastingmoor}.

\subsection{Seasonality}
In this section the seasonality of the consumption data is discussed. In \cite{Hoverstad2015}it was concluded that all the forecasting algorithms that were considered, produced more accurate forecasts when they were combined with a preprocessing stage that extracted the seasonality before forecasting, compared to applying the same algorithms directly on raw data. The forecasting model is left with the task of modelling the deviation of the template consumption instead which is less challenging than performing a forecast out of the blue. These templates or filter are extracted from the consumption dataset by the use of equations \ref{eq:daily_filter} and \ref{eq:weekly_filter}. $ D $ and $ W $ gives respectively the number of days and weeks in the dataset. $\bar{y}_i$ and $\bar{y}_j$ gives the consumption of half an hour, averaged over respectively all days and weeks.   \textbf{read paper again...}

\begin{equation}\label{eq:daily_filter}
	\bar{y}_i = \frac{1}{D} \sum_{d=1}^D y_{di} i \in [1,48]
\end{equation} 

\begin{equation}\label{eq:weekly_filter}
	\bar{y}_j = \frac{1}{W} \sum_{w=1}^W y_{wj}  j \in [1,336]
\end{equation} 


Figure \ref{daily_filter} shows the daily filter in appendix \ref{app:A}. 

Figure \ref{weekly_filter} show the weekly filter.



% plot the moving average of the year. Clearly see the impact of the summer and winter.
% This is a trend that can be taken into account when predicting.


In the daily and weekly filters there can clearly be seen a consumption peek after midnight. This is due to heat storage systems that use electricity in the hours of low tariff and that release heat during high electricity tariffs. 


\subsection{Influence of temperature}
% notes on correlation see OneNote.
% use the correlations. https://realpython.com/numpy-scipy-pandas-correlation-python/
% resample the consumption to daily and then apply some of the correlation techniques. 
In following section the correlation between the temperature and the electricity consumption is discussed.\\


\textbf{Pearson correlation}\\
The Pearson correlation is a measurement of the linear dependency between two variables which is based on the covariance variable. A Pearson correlation values gives information concerning the magnitude of the association and the corresponding direction of it. A Pearson value of one and minus one give respectively a perfect positive and negative linear relation between the variables. A value of zero, corresponds to independent behaviour. Following formula is used when calculating the Pearson correlation. 

\begin{equation}\label{eq:pearson}
	\rho_{X,Y} = \frac{\sigma_{x,y}}{\sigma_x\sigma_y}
\end{equation}

Assumptions concerning Pearson correlation are that samples used for the correlation should be independent, normal distributed and linear related to each other. Also, homoscedasticity is assumed. Homoscedasticity is important when performing linear regression and assumes that $ \sigma_x $ and $ \sigma_y $ are constant and not in function of each other. This final assumption is validated by making use of Figure \ref{fig:pearson}.

\begin{figure}[h!]
	\centering
	\includegraphics[width=0.8\textwidth]{pearson.png}
	\caption{Relation between consumption and temperature.}
	\label{fig:pearson}
\end{figure}

This figure shows the classic cone-shaped pattern of heteroscedasticity. On days when it is warm there is overall similar human behaviour in lowering the electricity consumption. However, on colder days the variation in consumption is higher. Because the assumptions of the Pearson correlation are not fulfilled, care should be taken with its output.\\

Applying the Pearson correlation on Figure \ref{fig:pearson} gives a correlation value of $ -0.85 $. This means there is a reasonable linearly decreasing relation.\\


\textbf{Spearman correlation}\\
Spearman correlation is a ``Rank correlation''. This means that the ordering of the consumption and temperature in a sample are each compared in their corresponding array of measurements.  When the ordering of both variables in a sample are similar, correlation is strong and positive. If the ordering is reversed, correlation is strong and negative. There is a perfect positive ordering if larger consumption always corresponds to a higher temperature. Notice that for a perfect ordering, no linear relation of the variables is necessary. The Spearman correlation coefficient is calculated using equation \ref{eq:pearson}, but takes into account the rank of a variable in all the measurements of this variable instead of the measurement value itself.\\

In order to use the spearman correlation data has to be ordinal, which means that it can be ordered. The spearman correlation gives information about the monotonicity relation between the variables. $ \rho = 1 $ corresponds to a monotonically increasing relation.\\

Applying the Spearman correlation  gives a correlation value of $ -0.87$, which means there is a reasonable negative monotonicity relation.\\

\textbf{Kendal correlation}
The ``Kendal correlation'' is also a rank based correlation. Here it is looked at the pairs of observation that are concordant, discordant or neither. A correlation coefficient close to one occurs when both variables have the same ranking and similar a coefficient close to minus one occurs when rankings in one variable are the reverse of the other. Equation \ref{eq:kendall} gives the equation to calculate the ``Kendal correlation coefficient''.

\begin{equation}\label{eq:kendall}
	\tau = \frac{n^+-n^-}{\sqrt{(n^++n^-+n^x)(n^++n^-+n^y)}}
\end{equation}
\begin{itemize}
	\item $ n^+ $ is the number of concordant pairs
	\item $ n^- $ is the number of discordant pairs
	\item $ n^x $ is the number of ties only in x
	\item $ n^y $ is the number of ties only in y
	\item concordant $\rightarrow $ $ (x_i > x_j ) $ and $ (y_i > y_j ) $ or $ (x_i < x_j ) $ and $ (y_i < y_j ) $
	\item discordant $\rightarrow $ $ (x_i > x_j ) $ and $ (y_i < y_j ) $ or $ (x_i < x_j ) $ and $ (y_i > y_j ) $
	\item neither $\rightarrow $ $ (x_i = x_j ) $ or $ (y_i = y_j ) $
	\item if both $ (x_i = x_j ) $ and $ (y_i = y_j ) $ $\rightarrow $ not included in either $ n^x $ or $ n^y $
\end{itemize}

Applying the Kendal correlation  gives a correlation value of $ -0.66$, which means there is a reasonable negative monotonicity relation.\\

% There is a clear dependency between temperature and electricity consumption, which means that electricity is used for heating.  


\subsection{Comparing weekdays with weekends}



\subsection{Impact of holidays}
% important that look at holidays in the UK. In paper \cite{Hoverstad2015} all the holydays are subsitued by the same day the next week and the previous week. It is possible to look at all the holidays, normalize them concerning temperature and try to get a seasonality model. 



\section{ARIMA}
% Idea is to use the simple ARIMA model as a base line forecasting model. 
% see datacamp and youtube Lola
% ARIMA assumes stationary data.
What is ARIMA. 
Assumptions of ARIMA...

\textbf{Stationarity}\\
 https://machinelearningmastery.com/remove-trends-seasonality-difference-transform-python/
When data is modelled it is assumed that the statistics of the data are consistent or stationary. This means the mean and standard deviation is not changing in time. However, because time series are often subdued to a trend or seasonality this assumption of stationarity is violated. In order to model not stationary observations by a stationary model as ARIMA, trends and seasonal effects should be removed. A way to check the stationarity of your observations, the ``Dicky-Fuller test'' can be used.
A way to remove non-stationarity is by using ``Difference Transform''. Here the trend and seasonality is subtracted from the observations leaving behind a stationary dataset.


\section{Conclusion}
The final section of the chapter gives an overview of the important results
of this chapter. This implies that the introductory chapter and the
concluding chapter don't need a conclusion.




%Please don't abuse enumerations: short enumerations shouldn't use
%``\verb|itemize|'' or ``\texttt{enumerate}'' environments.
%So \emph{never write}: 
%\begin{quote}
%	The Eiffel tower has three floors:
%	\begin{itemize}
%		\item the first one;
%		\item the second one;
%		\item the third one.
%	\end{itemize}
%\end{quote}
%But write:
%\begin{quote}
%	The Eiffel tower has three floors: the first one, the second one, and the
%	third one.
%\end{quote}

%%% Local Variables: 
%%% mode: latex
%%% TeX-master: "thesis"
%%% End: 

\chapter{State of the art short-term residential load forecasting techniques}
\label{cha:State of the art short-term residential load forecasting techniques}
Forecasting the electrical load of the different individual households has a couple of challenges. There should be dealt with the missing values, as discussed in section \ref{s:missing_data}. Also, the different time-series are influenced by exogenous factors as weather conditions and the day of the year. The dependency on exogenous variables can be a very non-linear relation and can have different effects on different households. For example depending on a house has solar panels, the consumption could be altered much. Only three indications of the temperature are given on a daily basis. Some additional information is know of certain households, but this data is very incomplete. Next, the individual load series have a high volatility and uncertainty with respect to a load signal on transmission level which shows more consistent seasonality and straight forward dependency on weather and calendar variables. This is because the contingency of the individual load data is mitigated due to averaging out of the uncertainty. Ofcourse, the obvious disadvantage is that only forecasts on this aggregated level can be made which is not the goal of our investigation.\\ To tackle the high non-linearity that is inherent to residential load forecasting in literature often ``Neural Networks'' are used.
\textbf{See also paper TA2 --> aggregated vs individual forecasting.}

\section{Introduction to Neural Networks}\label{s:Introduction to Neural Networks}
A  standard multilayer feedforward neuralnetwork with locally bounded piecewise continuous activation function can approximate any continuous function to any degree of accuracy if and only if the network's activation function is not a polynomial, as stated by \textbf{Leshno et al} in \textbf{1993}. This theorem proofs that a ``universal approximator'' exists for continuous functions, but it lacks the recipe to construct it. In \cite{Nielsen2015} it is shown that a feedforward network with a single layer is enough to approximate any function by a specified accuracy if the hidden layer has the possibility to add an unlimited amount of hidden neurons in its layer. It is discussed that when a function is discontinuous, which means that it makes sudden, sharps jumps, it is not possible to approximate the function by any prescribed accuracy. However, in practise a continuous approximation is often good enough.\\

Neural networks are suitable of learning very non-linear mappings between inputs and outputs. The difference between ``Deep Neural Networks'' and ``Shallow Neural Networks'' is the amount of layers of neurons are used inside the network. These layers of neurons, that are not inputs or outputs are called ``hidden neurons''. Because a ``Deep Neural Network'' has a hierarchical layout of the different hidden layers, it not only learning features from the non-linear combinations of inputs, but uses other layers to learn features of combinations of features learned in lower hidden layers. This is possible because higher hidden layers get the outputs of lower hidden layers as input. As discussed in  \cite{Shi2018} due to this characteristic, deep learning is suitable to learn multiple uncertainties with differing sharing levels over different households e.g. the amount of sunshine. However, because of the higher expressiveness (and often the amount of the to learn parameters), a ``Deep Neural Network'' with respect to a ``Shallow Neural Network'', suffers more of overfitting as is discussed in section \ref{s:Problems}.

\subsection{MLP}
The simples configuration of deep networks are multilayer perceptrons and they are made up out of multiple fully connected layers of neurons. Figure \ref{fig:MLP} shows a MLP with one hidden layer.

\begin{figure}[h!]
	\centering
	\includegraphics[width=0.8\textwidth]{MLP.png}
	\caption{Figure of a MLP (source \cite{Czum2020}).}
	\label{fig:MLP}
\end{figure}

All layers are connected to the next layer by the means of an affine function together with a non-linear activation function represented by sigma as shown by equation \ref{eq:NN} with $ \textbf{L}^{(N)} $ the vector with outputs of the Nthe layer, $ \textbf{W}^(N) $ the Nth weight matrix and $ \textbf{b}^{(N)} $ the Nth bias

%It can be noted that the MLP network is invariant.

\begin{equation}\label{eq:NN}
	\textbf{L}^{N+1} = \sigma(\textbf{W}^{(N)}\textbf{L}^{N}+\textbf{b}^{(N)}).
\end{equation}


\subsection{CNN}
\textbf{See oneNote}

\subsection{RNN}\label{s:RNN}
A recurrent Neural Network is a specialized neural network to deal with sequential information. While traditional deep neural networks assume that inputs and outputs are independent of each other, the output of recurrent neural networks depend on the prior elements within the sequence. In order to take past information from previous inputs into account, a hidden variable $ h_t $ is used. By making use of this variable which makes a summary of the previous seen information, an exponential increase in the number of model parameters is avoided. Cited from \cite{bibid}: ``Hidden states are technically speaking inputs to whatever we do at a given step, and they can only be computed by looking at data at previous time steps''. Equation \ref{eq:RNN} shows how the previous hidden state and the current information are merged in the next hidden state with $ \textbf{X}^t\in \mathbb{R}^{d\times 1} $, $ \textbf{H}^t\in \mathbb{R}^{h\times 1} $, $ \textbf{W}_1\in \mathbb{R}^{h\times d} $, $\textbf{W}_2\in \mathbb{R}^{h\times h} $ and $ \textbf{b}\in \mathbb{R}^{h\times 1} $ 

\begin{equation}\label{eq:RNN}
	\textbf{H}^{t+1} = \tanh(\textbf{W}_1\textbf{X}^{t}+\textbf{W}_2\textbf{H}^{t}+\textbf{b}).
\end{equation}

The equation $\textbf{X}^t$ corresponds to one example at time step $ t $ with dimensionality $ d $. 
Also a deep RNN is possible, where multiple hidden state per time step are used. \\

\begin{figure}[h]
	\centering
	\includegraphics[width=1.0\textwidth]{RNN.png}
	\caption{Figure of the logical flow of a vanilla RNN with a hidden state (source: \cite{Czum2020}).}
	\label{fig:RNN}
\end{figure}

As was discussed in Section \ref{s:Introduction to Neural Networks} a standard neural network can act as a ``universal approximator'' when given enough hidden states. A similar result exist for a recurrent neural network which states that it is capable to approximate a sequence-to-sequence mapping to  an arbitrary accuracy as discussed in \cite{Hammer2000}. However, as discussed in \cite{Teuwen2019} even if expressiveness of the simple model is very powerful in theory, this doesn't indicate that such a representation can be learn in a reasonably amount of time from a dataset. As will be discussed in Section \ref{s:Problems}, the main drawback of the vanilla recurrent neural network is that it forgets fast, important information in function of the amount of time steps. When using ``backpropagation through time'' for updating the weights, the gradients that corresponds to inputs seen a lot of time steps ago will become very small due to the multiplication of small gradients over the time steps. Therefore, there contribution of updating the weights of the recurrent neural network will be very small and thus this information will be ``forgotten''.


\subsection{Difficulties \& Solutions of neural networks}\label{s:Problems}
Neural Networks have a high expressiveness but comes at the cost of overfitting and a vanishing gradient.
when the NN is learning from training data, every epoch the error between the input and output of the training exemples is reduced. In the beginning the generalization error reduces simultaneously with the generalization error. The generalization error is the error that the model makes on data that is not in the training set. However, on a certain point during the training the generalization error increases while the training error still decreases. This means that the model is no longer learning ``intelligent'' general rules and patterns in the data, but is just remembering the training data and will therefore not apply in general. This is often the case in a model with high expressiveness because the model is less pushed to make generalizations and has the ability to just to remember the training data. Solutions to overfitting can be regularization which includes the parameter norms as a cost in the objective function. Typical choices for resembling the size of a parameter are the $ L_1 $ and the $ L_2 $ norms. Other methods that can be used are: early stopping, dropout and pruning.

It should also be noted that the gradient can increase very much over the different time steps, which in literature is called gradient explosion. The solution strategy for this is applying gradient clipping by norm or by value. Gradient clipping by norm means that when the two norm of the gradient $ \bm{\xi} $ exceeds a threshold value $ \theta $, the two norm of the gradient is scaled to equal the threshold value. The mathematical formulation is given by equation \ref{eq:clipping}:

\begin{equation}\label{eq:clipping}
	\bm{\xi}= min(1,\frac{\theta}{\| \bm{\xi} \|})\times\bm{\xi}.
\end{equation}


The second problem is the vanishing gradient problem which originates because while using the backpropagation algorithm to calculate the gradient which is used in different update methods of the weights, the gradient is calculated at the end of the NN and propagated back using every time the previous calculated gradient values which exponentially decreases in function of the time steps. Therefore, at the first layers of the network, the gradient has become so small that the weights are almost not updated anymore. In a RNN setting this corresponds to having a short term memory which means that initial inputs that were presented to the NN are being forgotten. Mitigation strategies often proposed in literature are LSTM and GRU. Both techniques have in common that they can learn which data in the sequence is important and should be retained and which information can be thrown away. It is important to state that LSTM and GRU are not solving the vanishing gradient problem as explained in \cite{Teuwen2019}. The gradient is still exponentially decreasing, but the effect is less pronounced as can be seen for LSTM in Figure \ref{fig:grad_exp}. When the forget gate, that sits inside a LSTM cell outputs a value that is close to one, the exponential decay will have also a base close to one. $ \tau $ gives the number of epochs. Also, the complexity of the recurrent models grows linearly with the amount of time steps that are processed in the sequence. As discussed in \cite{Teuwen2019}, the amount of memory and calculation effort needed to do a gradient update also increases linearly with the amount of time steps. Memory and calculation load can be mitigated by making use of ``truncated backpropagation through time''. \\ \textbf{Can put further explanation in attachment --> see assignament ANN}.

\begin{figure}[h!]
	\centering
	\includegraphics[width=0.8\textwidth]{grad_exp.png}
	\caption{Exponential decrease of the gradient size of a simple RNN (red) or a LSTM (blue) (source: \cite{Teuwen2019}).}
	\label{fig:grad_exp}
\end{figure}


\subsection{LSTM}\label{s:LSTM}

\begin{figure}[ht]
	\centering
	\includegraphics[width=1.0\textwidth]{LSTM_cell.png}
	\caption{A LSTM cell that is repeated over time (source: \cite{Olah}).}
	\label{fig:LSTM_cell}
\end{figure}

As discussed in Section \ref{s:Problems} the LSTM is an updated version of the conventional RNN first proposed by \textbf{Hochreiter \& Schmidhuber} in $ 1997 $ to deal with the short term memory it suffers from. A LSTM can longer take important aspects of the presented time series into account while outputting a current prediction. To do this a LSTM makes use of three gates: forget gate $ \textbf{f}_t $, input gate $ \textbf{i}_t $ and an output gate $ \textbf{o}_t $. When comparing the three gates with equation \ref{eq:RNN}, it is clear that every gate is by itself a recurrent neural network, with the only difference that a sigmoid function is used instead of a hyperbolic tangent. The core concept of the LSTM is that it makes use of a memory cell that is passed on through the different time steps. The memory cell contains important information that is seen before in the data and should be taken into account at the current new output. The three gates can delete, write and read information from this memory cell. It can also be noted that equation \ref{LSTM_rnn} is exactly equal to the conventional RNN described by equation \ref{eq:RNN}. Equation \ref{LSTM_rnn} processes the hidden states $ H_t $ and the new input $ X_t $ to propose an update $ \tilde{\textbf{c}}_t $ to the previous memory cell. The input gate (Eq. \ref{LSTM_input}) decides what will be preserved of the proposal and actually updated. The forget gate (Eq. \ref{LSTM_forget}) decides what will be preserved from the original memory cell $ \textbf{c}_t $. When both the old memory cell and the proposal are pruned, they are combined to one new memory cell. This new memory cell is together with the output gate (Eq. \ref{LSTM_output}) used to output new hidden states. \\

In order to train a LSTM neural network there are considerably more parameters that have to be learned. There are now four different weight matrices for both the hidden states and the inputs. Because by this increase of weights also the expressiveness of the model has increased with respect to the vanilla recurrent neural network of Section \ref{s:RNN}. Therefore, overfitting of the data should be extra monitored. Further, it can also be noted when looking to the LSTM equations that when the parameter that determines the amount of hidden states this will have a higher effect on the calculation load of a LSTM than the vanilla RNN. This is similar when more inputs are added.


The LSTM equations are given as follows as they were found in \cite{Teuwen2019}:
\begin{equation}\label{LSTM_forget}
	\textbf{f}_{t} = \sigma(\textbf{W}_{fH}\textbf{H}_{t-1}+\textbf{W}_{fX}\textbf{X}_{t-1}+\textbf{b}_{f}),
\end{equation}
\begin{equation}\label{LSTM_input}
	\textbf{i}_{t} = \sigma(\textbf{W}_{iH}\textbf{H}_{t-1}+\textbf{W}_{iX}\textbf{X}_{t-1}+\textbf{b}_{i}),
\end{equation}
\begin{equation}\label{LSTM_output}
	\textbf{o}_{t} = \sigma(\textbf{W}_{oH}\textbf{H}_{t-1}+\textbf{W}_{oX}\textbf{X}_{t-1}+\textbf{b}_{o}),
\end{equation}
\begin{equation}\label{LSTM_rnn}
	\tilde{\textbf{c}}_{t} = \tanh(\textbf{W}_{cH}\textbf{H}_{t-1}+\textbf{W}_{cX}\textbf{X}_{t-1}+\textbf{b}_{c}),
\end{equation}
\begin{equation}\label{LSTM_memory}
	\textbf{c}_t = \textbf{f}_t\times\textbf{c}_{t-1}+\textbf{i}_t\times\tilde{\textbf{c}}_t,
\end{equation}
\begin{equation}\label{LSTM_next}
	\textbf{H}_t = \textbf{o}_t\times\tanh(\textbf{c}_t).
\end{equation}


\subsection{GRU}\label{s:GRU}

\begin{figure}[ht]
	\centering
	\includegraphics[width=0.6\textwidth]{GRU_cell.png}
	\caption{A GRU cell that is repeated over time (source: \cite{Olah}).}
	\label{fig:GRU_cell}
\end{figure}
A gated recurrent unit neural network is a newer, simplified version of the LSTM that deals with the short term memory problem of a vanilla recurrent neural network. It was introduced by \textbf{Cho et al.} in $ 2014 $. The LSTM is changed by merging the forget and input gate into an update gate. Also, the seperate memory cell and hidden states are combined. The different performance between the variations of the LSTM neural network are discussed in Section \ref{s:Performance results between different models}. The GRU equations are given as follows as was found in \cite{Teuwen2019}:

\begin{equation}
	\textbf{z}_{t} = \sigma(\textbf{W}_{zH}\textbf{H}_{t-1}+\textbf{W}_{zX}\textbf{X}_{t-1}+\textbf{b}_{z}),
\end{equation}
\begin{equation}
	\textbf{r}_{t} = \sigma(\textbf{W}_{rH}\textbf{H}_{t-1}+\textbf{W}_{rX}\textbf{X}_{t-1}+\textbf{b}_{r}),
\end{equation}
\begin{equation}
	\tilde{\textbf{H}}_{t}=\tanh(\textbf{W}_{HH}(\textbf{r}_t\times\textbf{H}_{t-1})+\textbf{W}_{HX}\textbf{X}_t+\textbf{b}_H),
\end{equation}
\begin{equation}
	\textbf{H}_t=\textbf{z}_t\times\textbf{H}_{t-1}+(1-\textbf{z}_t)\times\tilde{\textbf{H}}_t.
\end{equation}

\subsection{Performance results between different models}\label{s:Performance results between different models}
Paper \cite{Chung2014} conducts an empirical evaluation of the GRU and compares it with the older LSTM. It was found that it outperformed the vanilla RNN and attained similar performance as the LSTM on the task of polyphonic music modeling and speech signal modeling. 
According to \cite{Olah}, the next step in sequence modelling can be the use of attention models or grid LSTM's.

There exists a lot of variations of the LSTM neural networks. Paper \cite{Greff2017} discusses a large-scale analysis of eight LSTM variants on the tasks of: speech recognition, handwritting recognition and polyphonic music modelling. The hyperparameters of the models were optimized using a random search method. It was found that none of the assessed variants of the conventional LSTM architecture could significantly outperform the latter. However, it was stated that the LSTM variants in some occasions were able to simpify the LSTM and its calculation load and number of parameters without the lose of performance.  Next, it was found that the forget and output gate were the most crucial gates of the LSTM network. When one of the two was removed, a significant lose of performance occurred. From the hyperparameter search is was found that the \textit{learning rate} is the most important hyperparameter, followed by the \textit{network size}. The use of a momentum term, which takes previous values of the weights into account when updating, was found to be unimportant in their setting, of online gradient descent. 

An analysis of the hyperparameter interaction was made and it showed that it was quite small. The largest interaction could be found between the learning rate and the size of the network. Therefore, during the hyperparameter search parameters can be varied individually which can drastically reduce the amount of runs have to performed to get the effect on the model. 


It was found that the learning 

\section{Short-Term residential electrical load forecasting}
\textbf{Pooling paper}
%Things should get out of paper:
%Problem that was solved.
%Method
%Result
Classical ways to deal with uncertainty.\\
Residential electrical load series have a high amount of volatility and uncertainty due to the contingency of the electrical consumption. Classical ways to deal with this are discussed in \cite{Shi2018} and listed as follows:
\begin{enumerate}
	\item Clustering to group similar houses based on historic load or exogenous consumption driving variables. Because the load or driving variables are similar in a cluster, the variance of uncertainty is also decreased. However, performance is very dependent of the dataset. \textbf{But the uncertainty on the whole is reduced --> on single household stays the same!!}
	\item Aggregating the residential loads to cancel out the uncertainties. The aggregated signal will show more regular patterns which means that is easier to predict.The downside is that the aggregated forecast will do a poor job of serving as forecast for a household
	\item A spectral analysis e.g. wavelet analysis, Fourier transforms and empirical mode decomposition aim at seperating a load serie into a regular pattern, an uncertain signal and noise. Because the amount of regularity is low in a residential load serie, this method is infeasible.
\end{enumerate}



In this paper \cite{Shi2018} a novel pooling-based deep recurrent neural network is proposed which collects load profiles of neighbouring houses into a pool of training inputs. Pooling of neighbouring households historical loads to serve as input of the ``Deep Recurrent Neural Network'', is proposed to increase the data volume and diversity of load forecasting, which mitigates the effect of overfitting present in a DRNN. The idea is as quoted by \cite{Shi2018} to use the interconnected spacial information to compensate insufficient temporal information.Thereby, the pool of data allows to learn the correlations between neighbouring households and the shared uncertainties coming from external factors e.g. temperature.  Also, due to the pooling of different households during training the DRNN is able to learn common uncertainties. In paper \cite{Shi2018} pools consisting of $ 10 $ households are used. From the pool of inputs every epoch a randomly chosen batch of load signals are fed to the network. LSTM is applied to mittigate the short term memory of the RNN. Additionally, there is been made use of early stopping to further avoid overfitting. To implement early stopping there has been looked at the ``MSE'' for k iterations, obtained by cross-validation. When the variance of this sequence gets smaller than a specified variable, training stops. When the training ends, performance is tested on each household by using the learned network to perform a feed-forward prediction of the electrical load.\\

An overview of the different steps that were done during the proposed method are: data cleaning and preprocessing $\rightarrow$ data pooling $\rightarrow$ data sampling $\rightarrow$ data training $\rightarrow$ benchmarking.\\

Performance of the proposed method was finally evaluated based on a test set of the last $ 30 $ days and consisting out of : 
\begin{enumerate}
	\item performance of the proposed method with respect to Vanilla RNN, SVR and DRNN (without pooling)
	\item the effect of the neural network depth and pooling
\end{enumerate}

The proposed DRNN with pooling outperforms all other four methods based on following three metrics:

\begin{equation}\label{eq:RMSE}
	RMSE = \sqrt{\frac{\sum_{t=1}^{N}(\hat{y}_t-y_t)^2}{N}}
\end{equation}
\begin{equation}\label{eq:NRMSE}
	NRMSE = \frac{RMSE}{y_{max}-y_{min}}
\end{equation}
\begin{equation}\label{eq:MAE}
	MAE = \frac{\sum_{t=1}^{N}\abs{\hat{y}_t-y_t}}{N}
\end{equation}
\textbf{Actually LSTM network}
The amount of which the PDRNN outperformed the other methods can be seen in Table \ref{tab:pooling_result}. The effect of the depth of the DRNN and the pooling method is depicted in Figure \ref{fig:Shi2018_result}. It can be seen that without the pooling method the DRNN only benefits from extra layers till three are used. This is because from that point, overfitting will reduce the generalization capacity of the DRNN. With the pooling technique, extra layers stays beneficial. It can thus be concluded that introducing extra hidden layers is a good choice to model the non-linear relations, but this can only be done efficiently when overfitting is mittigated by the use of a pooling strategy.The RNN with pooling used for benchmarking consisted out of five layers and thirty hidden units in each layer.

\begin{figure}[h!]
	\centering
	\includegraphics[width=1\textwidth]{pooling_result.png}
	\caption{Results obtained in paper \cite{Shi2018} using the PDRNN method.}
	\label{fig:Shi2018_result}
\end{figure}

\begin{figure}[h!]
	\centering
	\includegraphics[width=1\textwidth]{Shi2018_result.png}
	\caption{Influence of the number of layers and the pooling method used in \cite{Shi2018}.}
	\label{fig:Shi2018_result}
\end{figure}

GRU (Gated Reset Update) or LSTM (Long Short Term Memory) can be implemented. They are both enhancements of the vanilla RNN which suffers from a vanishing gradient which causes it to to behave without a long term memory. In practise to know which one works often both are tried \cite{Teuwen2019}. Stochastic gradient descent means that the approximated gradient is calculated from a random subset of the available data instead from the entire dataset. \\

\textbf{Short-term Residential load forecasting based on LSTM RNN paper}\\
In \cite{Kong2019} it is chosen for a LSTM approach to forecast the complex temporal consumption pattern which characterises a single household electricity load. It is discussed that the diversity in the aggregated level of the individual electrical loads, smooths the daily load profile. This has as effect that the aggregated electrical load time-serie becomes more predictable, while a single household electrical load is more dependent on the human behaviour of its residents. This is substantiated by making use of a density based clustering technique where it was shown that the different daily consumptions of the aggregated signal could be described by one cluster and no outliers. An outlier means that a daily consumption could not be assigned to a cluster. On the other hand for individual time series the amount of outliers could range to over $ 80 $. To compare the consistency of different individual load signals the amount of outliers could therefore be used.\\
Because the residents daily routine is characterizing the household load so much, this is tried to be learned directly inside the LSTM RNN.\\
Inputs that are given to the LSTM are k past half hour load measurements, the time of when this measurements were taken, the day of the week of the measurements and if this day is a holiday or not. In table \ref{tab:LSTM_lit_result} the results are shown of the LSTM RNN method in comparison with other forecasting techniques. It can be noted that the proposed technique outperforms the rest based on the average performance of $ 29,808 $ individual forecasts of half an hour individual loads. Forecasting was performed on $ 69 $ different electrical loads coming from households in Australia.  However, for individual load series forecasting the MAPE minimization is also remarkable when considering its simplicity in comparison with LSTM. Next, it was concluded that learning methods that had good performance on aggregated time-series e.g. IS-HF and KNN, perform much worse when predicting individual loads. \\
Further, by making use of a regresion technique in function of the amount of outliers it is shown that LSTM and BPNN (Back-Propagation Neural Network) perform similar for, as previously discussed, consistent individual loads. The LSTM only starts to differentiate in performance when inconsistency grows. To conclude things that lack in \cite{Kong2019} are practical useful forecasts of a timespan of $ 24 $h instead of only half an hour and making use of a rule of thumb when parameter tuning. Hyperparameters that can be tuned in LSTM are: learning rate, lag variable, amount of hidden layers and the amount of hidden nodes.\\

\begin{figure}[h!]
	\centering
	\includegraphics[width=1\textwidth]{prev_appr.png}
	\caption{Different approaches tried in \cite{Kong2019} and their averaged performance of $ 29,808 $ individual forecasts of half an hour individual loads. }
	\label{tab:LSTM_lit_result}
\end{figure}

\textbf{CNN-LSTM paper}\\
In \cite{Kim2019} a novel technique is proposed which makes use of a convolutional neural network from which the outputs are given to a LSTM recurrent network after which a fully connected neural network  is used to produce the outputs. The purpose of the CNN is to extract the features that are the main drivers of energy consumption and to remove the noise that comes initially together with the raw inputs. The CNN is made up out of convolution layers and pooling layers and makes use of the ``ReLU'' activation function. The main purpose of a convolution layer is to extract features while the pooling layer reduces the number of parameters by making use of the ``max pooling principle''. Using the ``max pooling principle'' means taking the max value of each neuron cluster of the previous layer. As discussed in paper \cite{Kong2019} LSTM is suitable to alleviate the problem of a vanishing or exploding gradient which characterized a simple RNN. LSTM is able to preserve long-term memory by making use of memory states that is used in the calculation of hidden states. It is therefore suitable to remembering the irregular trend of the electrical load time-serie. Finally, a fully connected time-serie predicts the load forecast.\\
Paper \cite{Kim2019} further showed superiority with respect to only making use of the LSTM layers as can be seen in Table \ref{tab:CNN-LSTM_results}. The  Inputs that were used to forecast the household load which is located in France are: three submeters with historical loads, global intensity, voltage, global reactive power, global active power, time, data and month. 
At last, also an analysis is performed to investigate the influence of the different inputs by calculating the average class activation score over the inputs. The results are shown in Figure \ref{fig:LSTM-CNN_results}. It can be seen that especially ``Sub metering $ 3 $'' has a big influence on the final forecasts. This sub meter corresponds to the the electric water heater and air conditioner of the house. As was shown in Section \ref{tab:attributes} the dataset used in this thesis gives only information about the presence of a hot water heater.  Discussed limitations in the paper are the definition of the hyper parameters that were set by trail and error instead of using an automated method e.g. a genetic algorithm. A further limitation is the lack of household characteristics e.g. the amount of residents living in the house. It has previously been shown by \textbf{C. Beckel et al.} that household occupancy is one of the primarily drivers of electrical consumption in a household.\\

\begin{figure}[h!]
	\centering
	\includegraphics[width=1\textwidth]{CNN-LSTM_results_F.png}
	\caption{The importance of the different inputs as based on the average class activation score. (source \cite{Kim2019})}
	\label{tab:LSTM_lit_results}
\end{figure}

\begin{figure}[h!]
	\centering
	\includegraphics[width=1\textwidth]{CNN-LSTM_results_T.png}
	\caption{Comparison between LSTM and CNN-LSTM. (source: \cite{Kim2019})}
	\label{tab:LSTM_lit_results}
\end{figure}

\textbf{CNN-GRU paper}\\
\cite{Sajjad2020}

\textbf{See oneNote for the summary of the paper and say that it is showed that CNN-GRU performs even better than CNN-LSTM}


%\begin{table}
%  \centering
%  \begin{tabular}{||l|lr||} \hline
%    gnats     & gram      & \$13.65 \\ \cline{2-3}
%              & each      & .01 \\ \hline
%    gnu       & stuffed   & 92.50 \\ \cline{1-1} \cline{3-3}
%    emu       &           & 33.33 \\ \hline
%    armadillo & frozen    & 8.99 \\ \hline
%  \end{tabular}
%  \caption{A table with the wrong layout.}
%  \label{tab:wrong}
%\end{table}
%
%\begin{table}
%  \centering
%  \begin{tabular}{@{}llr@{}} \toprule
%    \multicolumn{2}{c}{Item} \\ \cmidrule(r){1-2}
%    Animal    & Description & Price (\$)\\ \midrule
%    Gnat      & per gram    & 13.65 \\
%              & each        & 0.01 \\
%    Gnu       & stuffed     & 92.50 \\
%    Emu       & stuffed     & 33.33 \\
%    Armadillo & frozen      & 8.99 \\ \bottomrule
%  \end{tabular}
%  \caption{A table with the correct layout.}
%  \label{tab:ok}
%\end{table}



\section{Conclusion}
The final section of the chapter gives an overview of the important results
of this chapter. This implies that the introductory chapter and the
concluding chapter don't need a conclusion.



%%% Local Variables: 
%%% mode: latex
%%% TeX-master: "thesis"
%%% End: 

\chapter{Clustering of the load profiles}
\label{cha:n}
Do a literature study about forecasting. What is the current state of the art methods to do forecasting. 

\section{The First Topic of this Chapter}
\subsection{Item 1}
\subsubsection{Sub-item 1}
\lipsum[80]

\subsubsection{Sub-item 2}
\lipsum[81]

\subsection{Item 2}
\lipsum[82]

\section{The Second Topic}
\lipsum[83-85]

\section{Conclusion}
\lipsum[86-88]

%%% Local Variables: 
%%% mode: latex
%%% TeX-master: "thesis"
%%% End: 

\chapter{Model evaluation}
\label{cha:Model evaluation}

This chapter discusses the models that were introduced in Chapter \ref{cha:Forecasting the daily electricity consumption} on the test set. As was shown in Table \ref{tab:summ_data}, the test set consists out of the days of the month December. Missing days in December are removed to avoid the influence of the estimation error of the reference signal. In this chapter first the model selection is explained after which a discussion of the performance on the test set follows.

\section{Model selection}\label{s:Model selection}
From Chapter \ref{cha:Forecasting the daily electricity consumption} the model parameters are tuned, but there is still a factor of random model performance due to the random initialization of the weight matrices. To reduce this influence, the model is trained $ 10 $ times and the model that performed best on a validation set is selected. As validation set the $ 10 $ last days of November are used. Also, during the training early stopping is performed wherefore an additional $ 10\% $ of the training data taken to serve as a second validation set. The patience parameter is taken as $ 5 $, which means that the validation error can increase $ 5 $ times before the model is stopped. The maximum amount of epochs that is allowed is $ 150 $. The values of the parameters for each of the three time series can be found in Chapter \ref{cha:Forecasting the daily electricity consumption}. Table \ref{tab:summ_model_selection} summarizes the results of the model selection. 

\textbf{Add here a table with the model selection results.}
- epochs
- loss on validation set




\subsubsection{Performance on the test set}
In this section the results on the test set are discussed of the model that was selected in Section \ref{s:Model selection}.

- The model, last ten days are used for getting best model run and 10\% of the training is used for early stopping. Then model that get after training is run on the test set. The neural models have the downside in comparison with the baseline models that they have to remove training data that is very close to the test set during training. For example, the base line models could use all the previous data to perform training till the desired day

For example model three, only uses training data till end October. Model one and model two use training data till $ 20 $ November and miss a random $ 10 \%$ of the data during the year. This loss of data was necessary for model tuning and the fact that the LSTM model make generalizations before they know the query i.e. the day to forecast (eager models) and the base models learn from the previous data when they know the query (lazy models).

Say that here use the models that obtained in the previous chapter. Make MSE plot for the different NN and the baseline models for each serie and make a comparison of the MAPE with the other baseline model. (by making use of barplots --> normalized with the worst performer) Performance on the test set. 

Make a plot of the day forecast for each NN model --> four graphs bundled. 
- for comparing performance in the same time serie --> mae is used. To compare performance between different time series --> MAPE is used. 
- for stateless random 10 percent is used as validation set. Remember that should shuffle the training set beforehand otherwise j
- for stateful --> the last 30 days of November will be used --> no shuffling is allowed.

- make a graph which shows all the different days that have to be forecasted on the x-axis and the MAE error on the y-axis for the different models. (line graph)

- give an indication how long the models trained --> amount of epochs. 

- when there are a lot of missing days in the test data: serie 1 (0 days) and the other two (8days) -->it is naturally that the model performs worse, the previous day it bases its forecast on is just an estimate. For the day that is forecasted are always days where the true reference signal is available. 



- bar plot for general performance (MAE and MAPE - basemodels and NN)

- line plot for the error on each day (MAE and MAPE - basemodels and NN)

- the signal for the forecast of a day for each model

- in some plots will see that there are gaps --> reason is that it is not useful to predict on the $ 8 $ missing days in the test set. There will only be an estimated signal available to calculate an error on. 

- the line plot shows the distribution of the error per day on the predictions per day. 

- also the mape metric is considered because it allows to compare between different series. 


\section{Conclusion}


%%% Local Variables: 
%%% mode: latex
%%% TeX-master: "thesis"
%%% End: 

\chapter{Conclusion}
\label{cha:conclusion}
The final chapter contains the overall conclusion. It also contains
suggestions for future work and industrial applications.

- say that because only use a small amount of simplistic inputs it found that after learning for a non-linear combination of previous consumptions the model was able to find a pattern in in the electricity consumption and make improvements with respect to the baseline models.

- stress the difficulty of the task --> not much informative data --> really a lot of uncertainty --> say that could identify a pattern in this data --> clear prediction of peaks.

- also look at comments of Swevers and build up of my previous thesis.

\section{Future work}
- because it was seen that much forecasts have the correct form with regard to the amount of peaks a first thing to try to solve this could be to adjust the error metric that is used during training from MSE to MAE. The peaks will be more proportional punished and this could lead that the model shifts down. 
- genetic algorithm to tune the parameters

- Track down the influence of each of the inputs.
- Do a more extensive parameter search --> not neglecting synergy between different regularizations. Better to simultaneously try all the different parameters.

%%% Local Variables: 
%%% mode: latex
%%% TeX-master: "thesis"
%%% End: 


% Indien er bijlagen zijn:
\appendixpage*          % indien gewenst
\appendix
\chapter{Introduction to the dataset}
\label{app:A}
Appendices hold useful data which is not essential to understand the work
done in the master's thesis. An example is a (program) source.
An appendix can also have sections as well as figures and references\cite{h2g2}.

\section{More Lorem}

\begin{figure}[h!]
	\centering
	\includegraphics[width=0.8\textwidth]{amountNaN.png}
	\caption{The amount of NaN values in all the 3248 smart meters.}
	\label{fig:amountNaN}
\end{figure}
\subsection{Lorem 15--17}

\begin{table}[h]
	\centering
	\begin{tabular}{|p{5cm}|p{2.5cm}|}
		\hline
		\textbf{Attribute} & \textbf{Filled places}\\ \hline	
		Dwelling type  & 1702\\ \hline
		\# Occupants & 74\\ \hline
		Heating fuel & 1859\\ \hline
		Heating fuel & 78\\ \hline
		Hot water fuel & 76\\ \hline
		Boiler age & 74\\ \hline
		Loft insulation & 75\\ \hline
		Wall insulation & 75\\ \hline
		Heating temperature & 74\\ \hline
		Efficient lighting percentage & 73\\ \hline
		Dishwasher & 76\\ \hline
		Freezer & 70\\ \hline
		Fridge freezer & 70\\ \hline
		Refrigerator & 73\\ \hline
		Tumble Dryer & 76\\ \hline
		Washing machine & 76\\ \hline
		Game console &72\\ \hline
		Laptop & 70\\ \hline
		Pc & 70\\ \hline
		Router & 69\\ \hline
		Set top box & 70\\ \hline
		Tablet & 70\\ \hline
		Tv & 75\\ \hline
		
	\end{tabular}
	\caption{Amount of response on the voluntary questionnaires. }
	\label{tab:attributes}
\end{table}


\subsection{Lorem 18--19}


\section{Lorem 51}


%%% Local Variables: 
%%% mode: latex
%%% TeX-master: "thesis"
%%% End: 

\chapter{Forecasting the daily electricity consumption - extra}
\label{app:B}

In this appendix extra information and Figures are added that are not necessary to understand the work discussed in Chapter \ref{cha:Forecasting the daily electricity consumption}.

\section{Baseline models}

 \begin{figure}
	\centering
	\includegraphics[width=1.0\linewidth]{histogram_mape.png}
	\caption{An example histogram of the consumption in [kWh] versus count [-] used during MAPE forecast.}
	\label{fig:histogram_mape}
\end{figure}





%%% Local Variables: 
%%% mode: latex
%%% TeX-master: "thesis"
%%% End: 

\chapter{Extensions on the evaluation results}
\label{app:Extensions on the evaluation results}

\section{Results on the testset}

\begin{figure}[h]
	\centering
	\includegraphics[width=0.8\linewidth]{MAE_1_line.png}
	\caption{The MAE performance for the different days in the test set for Serie 1.}
	\label{fig:MAE_line_serie1}
\end{figure}

\begin{figure}[h]
	\centering
	\includegraphics[width=0.8\linewidth]{MAE_2_line.png}
	\caption{The MAE performance for the different days in the test set for Serie 2.}
	\label{fig:MAE_line_serie2}
\end{figure}	

\begin{figure}[h]
	\centering
	\includegraphics[width=0.8\linewidth]{MAE_3_line.png}
	\caption{The MAE performance for the different days in the test set for Serie 3.}
	\label{fig:MAE_line_serie3}
\end{figure}









%\section{Old stuff}
%\subsection{Removing outliers}


% some meters don't have a lot of missing values, but have very untraditional output. Two cases are looked into
% 1. big deviation from the average meter. 
% 2. a full day of zeros is included. 
% 3. the moving average changes spectaculair --> fundamental change in the energy consumption
% whitch is hard to forecast for. (not consistent with a normal consumption pattern)
% Weird meters that are identified: 2985, 2984,
% Normal meters: 2979, 2982
% idea to check also for outliers on monthly/weekly scale? 
%After the missing values are replaced by estimations, the outliers of the electricity consumption signals are identified.
%This is done by looking  at the z-scores of the yearly consumptions. A z-score is calculated using equation \ref{eq:z-score} and assumes that the yearly consumptions are normally distributed around the average consumption. Consumptions that have a very low probability to occur are removed by imposing that $ |z-score| < 3 $.

%\begin{equation}
%	z-score = \frac{x-\mu}{\sigma}
%\end{equation}                      
%
%Figure \ref{fig:z-score} gives the obtained z-values. It can be seen that $ 6 $ meters with an unlikely high or low consumption are removed. 
%
%\begin{figure}[h!]
%	\centering
%	\includegraphics[width=0.8\textwidth]{z-score.png}
%	\caption{Z-scores calculated from the yearly consumptions.}
%	\label{fig:z-score}
%\end{figure}


%\subsection{Normalization of the data}
%% normalize as done in ppt --> deviding by the yearly consumption.
%% downside of this normalilzation that outshooters will have influence.
%Normalization is necessary because while absolute consumption differs, relative patterns of human behaviour are more similar \cite{Lago2020}. The patterns in the human behaviour is what a forecasting model is trying to predict and normalization contributes by avoiding the disturbance of different magnitudes in which this human pattern may occur. Every individual household time-serie is normalized based on its maximum and minimum value according to equation \ref{eq:norm}. 
%
%\begin{equation}\label{eq:norm}
%	normalized value = \frac{x - x_{min}}{x_{max} - x_{min}}
%\end{equation}  
%
%As discussed in section \ref{s:Basic analysis} the average is taken over all the normalized time-series to obtain a single signal.\textbf{Ask if this is good??}  Because the maximum is taken into account during the normalization, measurement out shooters have an influence on the normalization. 


%\section{ARIMA}
%% Idea is to use the simple ARIMA model as a base line forecasting model. 
%% see datacamp and youtube Lola
%% ARIMA assumes stationary data.
%What is ARIMA. 
%Assumptions of ARIMA...
%
%\textbf{Stationarity}\\
%https://machinelearningmastery.com/remove-trends-seasonality-difference-transform-python/
%When data is modelled it is assumed that the statistics of the data are consistent or stationary. This means the mean and standard deviation is not changing in time. However, because time series are often subdued to a trend or seasonality this assumption of stationarity is violated. In order to model not stationary observations by a stationary model as ARIMA, trends and seasonal effects should be removed. A way to check the stationarity of your observations, the ``Dicky-Fuller test'' can be used.
%A way to remove non-stationarity is by using ``Difference Transform''. Here the trend and seasonality is subtracted from the observations leaving behind a stationary dataset.


%%% Local Variables: 
%%% mode: latex
%%% TeX-master: "thesis"
%%% End: 


\backmatter
% Na de bijlagen plaatst men nog de bibliografie.
% Je kan de  standaard "abbrv" bibliografiestijl vervangen door een andere.
\bibliographystyle{abbrv}
\bibliography{Time_Series}

\end{document}

%%% Local Variables: 
%%% mode: latex
%%% TeX-master: t
%%% End: 
