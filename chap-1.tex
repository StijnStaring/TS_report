\chapter{Basic data analysis}
\label{cha:1}
In this chapter details of the dataset are introduced and a basic analysis is performed. This includes assessing missing data, seasonality,  influence of temperature and household data, comparing weekdays and weekends, applying an ARIMA model for forecasting.

% Look at the comments during the meeting 'writing the WTK thesis'.
% Make sure that write every part sufficient substantiated.

\section{Introduction to dataset}
% how the dataset is made up --> use the information of the competition.


\section{Missing data}
First comes the introduction to this topic.

\section{Seasonality}


\section{Influence of temperature}


\section{Influence of household data}


\section{Comparing weekdays with weekends}

\section{Impact of holidays}


\section{ARIMA}


\section{Conclusion}
The final section of the chapter gives an overview of the important results
of this chapter. This implies that the introductory chapter and the
concluding chapter don't need a conclusion.




%Please don't abuse enumerations: short enumerations shouldn't use
%``\verb|itemize|'' or ``\texttt{enumerate}'' environments.
%So \emph{never write}: 
%\begin{quote}
%	The Eiffel tower has three floors:
%	\begin{itemize}
%		\item the first one;
%		\item the second one;
%		\item the third one.
%	\end{itemize}
%\end{quote}
%But write:
%\begin{quote}
%	The Eiffel tower has three floors: the first one, the second one, and the
%	third one.
%\end{quote}

%%% Local Variables: 
%%% mode: latex
%%% TeX-master: "thesis"
%%% End: 
