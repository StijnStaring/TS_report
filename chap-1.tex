\chapter{Exploratory Data Analysis}
\label{cha:Data analysis}
This chapter starts by describing the dataset were the exploratory data analysis will be performed on in Section \ref{s:Data description}. Next, the preprocessing steps done before the data analysis are explained in Section \ref{s:Preprocessing}. The preprocessing steps consists out of handling missing data, identifying zero days, normalization and removing time series with a big shift in its rolling mean for all the 261 load series with a full year of smart meter measurements. In Section \ref{s:Data Analysis} follows a data analysis on the preprocessed data. For this all the 261 load series with a full year of smart meter measurements are aggregated to identify general characteristics of the data. Things of the aggregated load serie that are assessed are seasonality, comparing electrical consumption between weekdays and weekends, impact of an holiday, the influence of the temperature and the identification of the influence of properties of the household e.g. dwelling type. This last part was possible due to the availability of extra information through a voluntary questionnaire.


\section{Data description}\label{s:Data description}
The data used in this thesis was made available for the \href{https://ieee-dataport.org/competitions/ieee-cis-technical-challenge-energy-prediction-smart-meter-data}{IEEE-CIS technical challenge on energy prediction from smart data}. The dataset consists of load signals with time steps of 30 minutes of 3248 households located in the UK during the year 2017. Only households are considered and no The definition of an household are all the people who occupy a single housing unit, regardless of their relationship to one another. Each smart meter is property of E.ON UK and can collect a maximum  of $17520$ measurements during the year 2017.
 Not all the $3248$ smart meters consist out of full data as can be seen in Figure \ref{fig:amountNaN}. It can be clearly seen that there are $12$ jumps in the amount of missing values. This is because the available data ranges from one month (only December) to a full year of data. This acknowledges that customers may have joined the measuring campaign at different times during the year. There are additional missing values in the time series due to sending or receiving errors of the smart meter.\\
 
 \begin{figure}[h!]
 	\centering
 	\includegraphics[width=0.8\textwidth]{amountNaN.png}
 	\caption{The amount of NaN values in all the 3248 load signals.}
 	\label{fig:amountNaN}
 \end{figure}

Besides of the electricity consumption of the different households, also information is available about the average, minimum and maximum temperatures on a daily resolution. Finally, extra household information has been partially collected about 2143 households through voluntary surveys. This concerns the dwelling type, number of occupants, number of bedrooms etc. as further detailed in Table \ref{tab:attributes}. From the questionnaire it can be derived that the maximum amount of residents is 4 and there is a maximum amount of bedrooms of 5. The kind of house units are: flat, bungalow, detached house, semi detached house and terraced house. Industrial loads or small businesses e.g. a bakery is not considered. The available datasets and their features are summarized by Table \ref{tab:available_data}.\\

\begin{table}[h]%\[!htb\]	
	\raggedright
	\begin{tabular}[t]{@{}ll}
		\firsthline
		\textbf{Consumption.csv}&\\ \hline
		\# households &3248 \\ 
		Information & Electric load\\
		Max measurements/serie & 17520\\
		Granularity&$ 1/2$ hour\\ 
		Timespan&year 2017 \\    
		Location&UK\\ \bottomrule   
	\end{tabular}
	\hfill
	\raggedleft
	\begin{tabular}[t]{@{}ll}
		\firsthline
		\textbf{Weather.csv}&\\ \hline
		Information & Average temperature\\
		& Max temperature\\
		& Min temperature\\
		Granularity& daily\\ \hline
		\textbf{addInfo.csv}&\\ \hline
		\# households &2143 \\ \bottomrule		    
	\end{tabular}\\
	\caption{Summary of the available csv files form the IEEE-CIS technical challenge.}
	\label{tab:available_data}
\end{table}




\section{Preprocessing}\label{s:Preprocessing}

Following sections describe the preprocessing applied on the 261 load series containing measurements for the entire year. The preprocessing done here is in preparation of the data analysis of Section \ref{s:Data Analysis}. It must not be confused with the preprocessing done for the three considered load series in Chapter \ref{cha:Forecasting the daily electricity consumption}.

\subsection{Missing data} \label{s:missing_data}
As discussed in Section \ref{s:Data description}, there exists two types of missing data in the Consumption csv file as was stated in the data description of the competition: fully missing months, due to the later participation in the measuring campaigns and missing values due to sending or receiving errors of the smart meter. When a smart meter fails, always all the measurements of that day are lost. In this Section two methods to impute the missing values are compared. Method Average Neighbours: replaces a missing value by the mean of the consumption values at the same moment on the next and previous days. Method Mean: substitutes the missing values of a time serie by the mean of all the measurements done by the meter. If the next or previous day is also missing in the serie, two days forward or back in time are used to replace the unknown day and so on. The resulting imputed signals can be seen in Figure \ref{fig:missing_values_imputing}. \\

\begin{figure}[h]
	\begin{subfigure}{0.5\textwidth}
		\includegraphics[width=1\linewidth]{mv_mean.png}
		\caption{Mean}
	\end{subfigure}	
	\begin{subfigure}{0.5\textwidth}
		\includegraphics[width=1\linewidth]{mv_s.png}
		\caption{Average Neighbours}
	\end{subfigure}
	\caption{Resulting time serie of the month March after imputation of the missing data.}
	\label{fig:missing_values_imputing}
\end{figure}

In order to compare how accurate both methods impute missing values, 181 months of March without any missing values were used and in each month randomly 7 days were removed. After applying both imputation methods, the resulting estimated signal was compared with the original one and an error value between both signals is calculated, using the mean squared error metric. Normalization is done by dividing the calculated MSE's by the MSE of the worst performing method to calculate the percentage of improvement of one method in comparison to the other. Figure \ref{fig:mv_result} shows that on average, a reduction of the MSE of more then $ 20\% $ is achieved when the average neighbours method is used in comparison to the mean method. Therefore, the average neighbours method will be applied to impute the missing values of the 261 time series with a full year of smart meter measurements. The only exception is made when the first of January and thirty-one December are imputed. Because not both neighbouring days can be known, the mean method is used.

\begin{figure}[h!]
	\centering
	\includegraphics[width=0.8\textwidth]{mv_result.png}
	\caption{Resulting month of March after substitution of the missing values by the mean value of the measurements. }
	\label{fig:mv_result}
\end{figure}

%Actually, used the missing values a bit at random. First imputed the missing values using the average neighbours method and later during the forecasting phase, used different imputation techniques that were not compared with the initial imputation techniques.


\subsection{Zero days}

When inspecting the load series, some untraditional meter measurements were identified. There were $ 9 $ meters of the 261 that had more than one day with a total electrical consumption of zero. Because it is unlikely that a household produces exactly zero kWh on an entire day all these $ 9 $ meters were removed. One of 9 load series is compared in Figure \ref{fig:zero_con} with a load serie with zero days and it is clear that the horizontal lines are not part of a normal household load signal.

\begin{figure}[ht]
	\begin{subfigure}{0.49\textwidth}
		\vspace{4mm}
		\includegraphics[width=1\linewidth]{zero_days.png}
		\caption{Serie with more than one day with zero consumption.}
	\end{subfigure}	 	
	\begin{subfigure}{0.49\textwidth}
		\includegraphics[width=1\linewidth]{without_zero_days.png}
		\caption{Serie $ 2 $ from Chapter \ref{cha:Forecasting the daily electricity consumption}.}
	\end{subfigure}	
	\caption{Comparison of series with and without zero load days.}
	\label{fig:zero_con}
\end{figure}

\subsection{Normalization of the data}
\label{s:Normalization of the data}
% normalize as done in ppt --> deviding by the yearly consumption.
% downside of this normalilzation that outshooters will have influence.
Normalization is necessary because while absolute consumption differs, relative patterns of human behaviour can be more similar according to \cite{Lago2020}. The goal of a forecasting model applied on an individual household is to extract the human behaviour and normalization contributes to this by avoiding the disturbance of different magnitudes in which this human pattern may occur. Every load serie is normalized based on its yearly consumption as was done by \cite{Lago2020}. The advantage of using the yearly consumption for normalization in comparison to the min-max method often used in literature, is the robustness against measurement outliers. This is important because in Section \ref{s:Data Analysis} all the 261 time series will be aggregated, which means that when a serie has only one very large outlier, the rest of its consumption values will be observed as small values in comparison with other series without a very large outlier. This is avoided when using the yearly consumption normalization where every load serie will be one at the end of the year as shown by Eq. \ref{eq:norm}. This allows for better comparison among the different series. The min-max normalization method is however used in Chapter \ref{cha:Forecasting the daily electricity consumption}, but here the series are always assessed individually

\begin{equation}\label{eq:norm}
	normalized\hspace{0.3cm} consumption_i = \frac{consumption_i}{\sum_{k=1}^{17520} consumption_k}.
\end{equation} 


\subsection{Shifts in rolling mean of load signal} \label{s:Shifts in rolling mean of load signal}
In this Section the normalized time series are assessed on fundamental changes in the load signal which can't be explained by normal human behaviour in the current household setting. A fundamental change of the load signal can be caused by an extra inhabitant or when systems are installed during the year that use a lot of electricity e.g. air-conditioning. A fundamental change is identified by looking at the maximum difference of the maximum and minimum rolling mean consumption over $ 7 $ days. Figure \ref{fig:fund_change} shows all the maximum differences between the maximum and minimum weekly rolling averages for the 261 load series. Outliers in the the maximum difference are detected as is done in a boxplot, namely from the third quartile a distance of one and a half times the interquartile range is added and values higher are considered as outliers. The red line in Figure \ref{fig:fund_change} shows when a maximum difference is considered as an outlier. Finally, the outliers which corresponds to 5 load series above the red line, are removed because they are seen as a disturbance when the load signals are aggregated in Section \ref{s:Data Analysis}. Figure \ref{fig:rolling_mean_shift} shows one of the removed load signals. 

\begin{figure}[h!]
	\centering
	\includegraphics[width=0.8\textwidth]{fund_change.png}
	\caption{The maximum differences between the maximum and minimum weekly rolling mean for all the 261 different load signals.}
	\label{fig:fund_change}
\end{figure}


\begin{figure}[h!]
	\centering
	\includegraphics[width=0.8\textwidth]{fundamental_change.png}
	\caption{Removed load signal with a shift in the rolling mean.}
	\label{fig:rolling_mean_shift}
\end{figure}


\section{Data Analysis}\label{s:Data Analysis}
In this section the remaining $256$ time series are converted to a single load serie by taking the mean. The single load serie is further referenced as the mean signal. This is done to identify general characteristics of the data. Things that are going to be assessed are: seasonality, comparing electrical consumption between weekdays and weekends, impact of an holiday, the influence of the temperature and the influence of properties of the household e.g. dwelling type.


% An individual household consumption time-serie is much subdued to complex and personal decisions that cause increases or decreases of the consumption. It is hard to capture all theses effects in a single model. By aggregation of the individual time-series by taking the average, this noisy individual behaviour is mitigated. The aggregated signal is now modelled and the increase or decrease of the consumption can be explained by a small set of variables. The aggregated signal can be seen as a ``virtual distribution substation'' as discussed in \cite{Hoverstad2015}. 
 
\subsection{Seasonality}
In \cite{Hoverstad2015} it is concluded that all the forecasting algorithms that were considered, produced more accurate forecasts when they were combined with a preprocessing stage that extracted the seasonality before forecasting, compared to applying the same algorithms directly on raw data. The forecasting model is left with the task of modelling the deviation from the template consumption instead of performing a forecast out of the blue. However in \cite{Hoverstad2015} they made forecasts of an aggregated signal which had a reasonable amount of regularity which is not the case for electrical consumption forecasting of individual households. That it is not useful to extract a regular pattern is in accordance to \cite{Shi2018}, where it is explained that the use of a spectral analysis such as a wavelet analysis, that aims at separating the regular pattern from the uncertainty and the noise, is not applicable during load forecasting of individual households due to the low amount of regularity. However, the analyses of the seasonality of the mean signal is still informative to get a feeling of the general human behaviour.\\

These day and week templates are extracted from the mean signal by the use of equations \ref{eq:daily_filter} and \ref{eq:weekly_filter} that calculate respectively the average day and week indicated by a thick blue line in Figure \ref{fig:average_signals}.

\begin{equation}\label{eq:daily_filter}
	\bar{y}_i = \frac{1}{D} \sum_{d=1}^D y_{d,i}, \hspace{10mm} i \in [1,48],
\end{equation} 

\begin{equation}\label{eq:weekly_filter}
	\bar{y}_j = \frac{1}{W} \sum_{w=1}^W y_{w,j}, \hspace{10mm}  j \in [1,336].
\end{equation} 

 $ D $ and $ W $ give respectively the amount of days and weeks in the year 2017. $\bar{y}_i$ and $\bar{y}_j$ give the consumption of half an hour, averaged over respectively all days and weeks. In Figure \ref{fig:average_signals} a clear consumption peak can be seen after midnight. This is due to heat storage systems that use electricity in the hours of low tariff and that release heat during high electricity tariffs. The daily seasonality shows a small peak in the load around 7 am and a bigger one around 6 pm. In the weekly seasonality it can be seen that all the peaks at 6 pm are of the same height but the smaller peak is different depending if it is a weekday or weekend as will be further explained in Section \ref{s:Comparing weekdays with weekends}. 

\begin{figure}[h!]
	\begin{subfigure}{1.0\textwidth}
		\centering
		\includegraphics[width=0.7\linewidth]{daily_filter.png}
		\caption{Daily seasonality}
	\end{subfigure}	 	
	\begin{subfigure}{1.0\textwidth}
		\centering
		\includegraphics[width=0.7\linewidth]{weekly_filter.png}
		\caption{Weekly seasonality}
	\end{subfigure}	
	\caption{The seasonality of the electrical load during the year 2017. The blue line indicates the average load signal. }
	\label{fig:average_signals}
\end{figure}


% plot the moving average of the year. Clearly see the impact of the summer and winter.
% This is a trend that can be taken into account when predicting.



\subsection{Comparing weekdays with weekends} \label{s:Comparing weekdays with weekends}
Weekdays vs weekends can be compared with the help of Figure \ref{fig:weekly_filter}. The reader is reminded that in order to get this graph, all the remaining household loads after preprocessing are averaged after which all the weeks are again averaged using equation \ref{eq:weekly_filter}. It can be seen that the consumption of the average business day is similar to a weekend day concerning the two main peaks during the day (7 am and 6 pm) and the sharp peak at midnight. However, it can be seen that the first peak during the day is higher and goes less down again during the weekend. This effect can be seen both during a Sunday and Saterday, but is most visible during a sunday. To proof previous statements similarity is measured by calculating the hourly difference of the $ 21 $ combinations that can be made of two different days. Figure \ref{fig:similarity_weekday} shows in blue and orange the error of combinations between business days or weekend days and in green the error of combinations between a business day and weekend day. The error value is calculated by summing the hourly errors between two days. It can be clearly seen that when a business day and weekend day are combined the error (green) is bigger and thus similarity smaller. Another thing that can be noticed is that the left cluster of green dots corresponds to a Saterday and the right to a Sunday. It can be noticed that Saterdays are more similar to a business day than a Sunday. 

\begin{figure}[h!]
	\centering
	\includegraphics[width=0.8\textwidth]{similarity_weekdays.png}
	\caption{Error between different pairs of weekdays.}
	\label{fig:similarity_weekdays}
\end{figure}




\textbf{Add here more specific how the error is calculated between holiday and weekday --> MAE and normalized.}

\subsection{Impact of holidays}\label{s:Impact of holidays}
% important that look at holidays in the UK. In paper \cite{Hoverstad2015} all the holydays are subsitued by the same day the next week and the previous week. It is possible to look at all the holidays, normalize them concerning temperature and try to get a seasonality model. 
In order to look at the impact of a holiday, all the holidays of the English and welsh holiday calendar are identified for the year $ 2017 $. For each of the $ 8 $ holidays a corresponding business day is selected with an as close as possible average temperature of the day. This is done to mitigate the temperature dependency. The resulting average holiday and business day is given in Figure \ref{fig:bvsh}. A holiday behaves similar to a weekend day with the first peak load going higher and goes less down over time. Figure \ref{fig:sim_weekdays} shows that a holiday behaves the most similar to a Sunday .

It can be seen that the consumption of a holiday behaves similar as a weekend day. Figure shows the average error between a holiday vs business day and a holiday vs weekend day. The error is calculated as discussed in section \ref{s:Comparing weekdays with weekends}. 

\begin{figure}[h!]
	\centering
	\includegraphics[width=0.8\textwidth]{bvsh.png}
	\caption{Figure with the comparison between holidays and business days.}
	\label{fig:bvsh}
\end{figure}

\begin{figure}[h!]
	\centering
	\includegraphics[width=1.0\textwidth]{sim_weekdays.png}
	\caption{Error between a holiday and other days of the week.}
	\label{fig:sim_weekdays}
\end{figure}


\subsection{Influence of temperature}
% notes on correlation see OneNote.
% use the correlations. https://realpython.com/numpy-scipy-pandas-correlation-python/
% resample the consumption to daily and then apply some of the correlation techniques. 
In following section the correlation between the temperature and the electricity consumption is discussed.\\


\textbf{Pearson correlation}\\
The Pearson correlation is a measurement of the linear dependency between two variables which is based on the covariance variable. A Pearson correlation value gives information concerning the magnitude of the association and the corresponding direction of it. A Pearson value of one and minus one give respectively a perfect positive and negative linear relation between the variables. A value of zero, corresponds to independent behaviour. Following formula is used when calculating the Pearson correlation

\begin{equation}\label{eq:pearson}
	\rho_{X,Y} = \frac{\sigma_{x,y}}{\sigma_x\sigma_y}.
\end{equation}

Assumptions concerning Pearson correlation are that samples used for the correlation should be independent drawn, come in pairs, follow homoscedasticity and there are no outliers. Outliers are especially undesirable when there are not a lot of samples. The variables should be normal distributed, linear related to each other and be continuous.\\
The samples used for the correlation are generated by calculating the daily consumptions matched with the daily average temperature. In this case the above assumptions are thus not valid. Homoscedasticity is important when performing linear regression and assumes that $ \sigma_x $ and $ \sigma_y $ are constant. This assumption is validated by making use of Figure \ref{fig:pearson}.

\begin{figure}[h!]
	\centering
	\includegraphics[width=0.8\textwidth]{pearson.png}
	\caption{Relation between normalized daily consumption and daily temperature.}
	\label{fig:pearson}
\end{figure}

This figure shows the classic cone-shaped pattern of heteroscedasticity. On days when it is warm there is overall similar human behaviour in lowering the electricity consumption. However, on colder days the variation in consumption is higher, which means that homoscedasticity is not fulfilled. Because the assumptions of the Pearson correlation are not fulfilled, care should be taken with its output.\\

Applying the Pearson correlation on Figure \ref{fig:pearson} gives a correlation value of $ -0.87 $. This means there is a reasonable linearly decreasing relation.\\


\textbf{Spearman correlation}\\
Spearman correlation is a ``Rank correlation''. This means that the ordering of the consumption and temperature in a sample are each compared in their corresponding array of measurements.  When the ordering of both variables in a sample are similar, correlation is strong and positive. If the ordering is reversed, correlation is strong and negative. There is a perfect positive ordering if larger consumption always corresponds to a higher temperature. Notice that for a perfect ordering, no linear relation of the variables is necessary. The Spearman correlation coefficient is calculated using equation \ref{eq:pearson}, but takes into account the rank of a variable in all the measurements of this variable instead of the measurement value itself.\\

In order to use the spearman correlation data has to be ordinal, which means that it can be ordered. The spearman correlation gives information about the monotonicity relation between the variables. $ \rho = 1 $ corresponds to a monotonically increasing relation.\\

Applying the Spearman correlation  gives a correlation value of $ -0.89$, which means there is a good negative monotone relation. This means if the temperature is higher, consumption is likely to be lower. Identically, if the temperature is lower it is likely that the consumption will be higher. \\

\textbf{Kendal correlation}
The ``Kendal correlation'' is also a rank based correlation. Here it is looked at the pairs of observation that are concordant, discordant or neither. A correlation coefficient close to one occurs when both variables have the same ranking and similar a coefficient close to minus one occurs when rankings in one variable are the reverse of the other. Equation \ref{eq:kendall} gives the equation to calculate the ``Kendal correlation coefficient''.

\begin{equation}\label{eq:kendall}
	\tau = \frac{n^+-n^-}{\sqrt{(n^++n^-+n^x)(n^++n^-+n^y)}}
\end{equation}
\begin{itemize}
	\item $ n^+ $ is the number of concordant pairs
	\item $ n^- $ is the number of discordant pairs
	\item $ n^x $ is the number of ties only in x
	\item $ n^y $ is the number of ties only in y
	\item concordant $\rightarrow $ $ (x_i > x_j ) $ and $ (y_i > y_j ) $ or $ (x_i < x_j ) $ and $ (y_i < y_j ) $
	\item discordant $\rightarrow $ $ (x_i > x_j ) $ and $ (y_i < y_j ) $ or $ (x_i < x_j ) $ and $ (y_i > y_j ) $
	\item neither $\rightarrow $ $ (x_i = x_j ) $ or $ (y_i = y_j ) $
	\item if both $ (x_i = x_j ) $ and $ (y_i = y_j ) $ $\rightarrow $ not included in either $ n^x $ or $ n^y $
\end{itemize}

Applying the Kendal correlation  gives a correlation value of $ -0.67$, which means there is a reasonable negative monotonicity relation.\\

% There is a clear dependency between temperature and electricity consumption, which means that electricity is used for heating.  



\subsection{Identification of driving attributes} \label{s:Identification of driving attributes}
In this section the influence of the extra knowledge about the kind of household where the smart meter is located, is investigated. This is not done by using a single averaged signal as was the case in the previous analysis sections. Now, every meter with additional information is considered. In Figure \ref{fig:bp_dwellingtype} the monthly consumption of the month December in function of dwelling type is shown. The month December is chosen, because this month is known for every smart meter. Missing values of the smart meters are substituted by method two, as discussed in section \ref{s:missing_data}. The amount of meters used for every visualization can be seen in Table \ref{tab:attributes}.

% Boxplots and consumptions are plotted. 

\begin{figure}[h!]
	\centering
	\includegraphics[width=1\textwidth]{bp_dwellingtype.png}
	\caption{Figure with the comparison of the different dwelling types.}
	\label{fig:bp_dwellingtype}
\end{figure}





Similar as was done in Figure \ref{fig:bp_dwellingtype} is also done for the other characteristics of the smart meters. The conclusions are listed below. As can be seen in Table \ref{tab:attributes}, some characteristics have not much data or the data is not much distribute over the different options of a characteristic. If this is the case, no reliable conclusions could be drawn. \textbf{Add concrete numbers!!}

\begin{itemize}
	\item There is a lot of variance in the monthly consumption of a detached house, but it has mostly a higher consumption than other dwelling types
	\item A ``real'' house (detached, semi-detached or terraced) tends to have higher monthly consumptions than a flat or bungalow.  
	\item The order of monthly consumption according to the mean and median values: Flat < Bungalow < Semi-detached < Terraced < Detached
	\item More occupants means more monthly consumption
	\item More rooms in the house means more monthly consumption
	\item Almost all houses use gas as heating fuel
	\item Almost all houses use gas as hot water fuel
	\item The age of the boiler has no clear effect on the monthly consumption
	\item The vast majority of the lofts are insulated
	\item The majority of walls are insulated
	\item The vast majority heats till a temperature between $ 18 $ and $ 20  $ degrees
	\item The majority of people has an efficient lighting percentage between $ 75\% $ and $ 100\% $
	
\end{itemize}



\section{Conclusion}
\textbf{Write a conclusion}
The final section of the chapter gives an overview of the important results
of this chapter. This implies that the introductory chapter and the
concluding chapter don't need a conclusion.




%Please don't abuse enumerations: short enumerations shouldn't use
%``\verb|itemize|'' or ``\texttt{enumerate}'' environments.
%So \emph{never write}: 
%\begin{quote}
%	The Eiffel tower has three floors:
%	\begin{itemize}
%		\item the first one;
%		\item the second one;
%		\item the third one.
%	\end{itemize}
%\end{quote}
%But write:
%\begin{quote}
%	The Eiffel tower has three floors: the first one, the second one, and the
%	third one.
%\end{quote}

%%% Local Variables: 
%%% mode: latex
%%% TeX-master: "thesis"
%%% End: 
