\chapter{Introduction}
\label{cha:intro}









\section{Importance of topic}
Individual household forecasting is a complex task because of the high amount of uncertainty
and the volatility of the data. To deal with this it was found in literature that often aggregated signals are forecasted instead. If there are papers that discuss electrical household consumption forecasting, they often use a lot of information about the household which will not be scalable in practice due to privacy concerns. This thesis investigates state of the art time series forecasting techniques based on LSTM neural networks that have as goal to forecast the next day of electrical household consumption, given only limited information.\\

When forecasting is improved on household scale, the customer can be better informed what the bill is going to be at the end of the month/year.
Energy producer can build a better trust with its customer by sending reliable bills. (Providing good service)
Producent can better estimate the energy demand of the whole customer population. This will lead to cheaper electricity production because a better planning is possible where there is less need of the more 
flexible but more expensive electricity installations e.g. diesel engines.


- see also table PL. 

- local prediction can be important for local changes to the net --> see oneNote goal PL. 

\section{Problem formulation and link with previous studies}
Now going to forecast individual houses, not aggregated signals. 


\section{Thesis objective and structure}
The goal of this thesis is to do short-term load forecasting for individual households. A forecast of the electrical load of a household for 24 hours. 

%%% Local Variables: 
%%% mode: latex
%%% TeX-master: "thesis"
%%% End: 
