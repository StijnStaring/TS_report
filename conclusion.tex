\chapter{Conclusion}
\label{cha:conclusion}
The final chapter contains the overall conclusion. It also contains
suggestions for future work and industrial applications.

- say that because only use a small amount of simplistic inputs it found that after learning for a non-linear combination of previous consumptions the model was able to find a pattern in in the electricity consumption and make improvements with respect to the baseline models.

- stress the difficulty of the task --> not much informative data --> really a lot of uncertainty --> say that could identify a pattern in this data --> clear prediction of peaks.

- also look at comments of Swevers and build up of my previous thesis.

\section{Future work}
- because it was seen that much forecasts have the correct form with regard to the amount of peaks a first thing to try to solve this could be to adjust the error metric that is used during training from MSE to MAE. The peaks will be more proportional punished and this could lead that the model shifts down. 
- genetic algorithm to tune the parameters

- Track down the influence of each of the inputs.
- Do a more extensive parameter search --> not neglecting synergy between different regularizations. Better to simultaneously try all the different parameters.

%%% Local Variables: 
%%% mode: latex
%%% TeX-master: "thesis"
%%% End: 
