\chapter{Conclusion}
\label{cha:conclusion}
This master thesis focussed on the implementation of deep LSTM neural networks to make 24 hours ahead predictions of the electrical consumption of individual households. The task of forecasting the electricity consumption of an individual household was seen to be of a significant complexity due to the high uncertainty and volatility in the load signal. This is because the load signal is directly influenced by the decisions made by the residents. The main contribution made with this thesis is showing that by only using the past electrical consumption behaviour, daily average temperature values and calendar data, it is possible with LSTM neural networks to outperform baseline models on 24 hours ahead forecasting, learn a non-linear relation between the inputs and amount of electrical consumption and predict peaks. This work aimed at implementing different versions of the LSTM neural network and to compare and investigate them.\\

The first part of this research elaborated the exploratory data analysis. It started with the description of the used dataset that originates from the IEEE-CIS technical challenge on energy prediction from smart data. Preprocessing on the series with a full year of measurements is carried out. Two methods to impute missing data are compared and it was found that the Average Neighbours methods was more suitable for estimating the missing data than the Mean method available in the Scikit-learn library. Further, consumption series with zero days and shifts in the rolling mean were identified and normalization was done using the total yearly consumption. Next followed a data analysis on an aggregated load signal. Seasonality was investigated for and the daily consumption profile showed a morning peak (07:30 am), an evening peak (06:00 pm) and a peak after midnight due to the use of heat storage systems. The consumption profiles of weekdays and weekend days are compared and it is observed that the morning peak is higher for a weekend day and decreases less. The influence of bank holidays is investigated and it was learned that a bank holiday behaves most similar to a Sunday. Also, there was a negative correlation between the temperature and the amount of electrical consumption. Finally, different household attributes were examined and it was seen that the order of dwelling types that are expected to consume more electricity is: Flat < Bungalow < Semi detached < Terraced < Detached. When there are more bedrooms in an housing unit, the consumption is also expected to be higher. \\ 

In Chapter \ref{cha:State of the art short-term residential load forecasting techniques} the conducted literature study is elaborated. First an introduction was given to neural networks and it was discussed that a vanishing gradient and overfitting are challenges for the vanilla recurrent neural network. A LSTM neural network can be used to mitigate the effect of a vanishing gradient and early stopping, adding the l2-norm of the weights in the objective function and dropout layers can be applied to mitigate overfitting. Next, it is discovered that the learning rate is a very important parameter to tune in a LSTM and different variants of a LSTM e.g. GRU can't significantly outperform the conventional LSTM architecture. The second part of the literature study discussed the state of the art short-term residential electrical load forecasting. First a pooling method was introduced that combined the load signals of neighbouring households in a pool of training data. Making use of pooling reduced the effect of overfitting. It was concluded that a deep LSTM is the most suitable model for short-term residential load forecasting with respect to ARIMA and SVR, and more LSTM layers can be beneficial. Further, a CNN on the raw data can extract features among several variables that effect electricity consumption and remove noise that comes initially together with the raw inputs. The use of a CNN-LSTM gave improved results with respect to a LSTM alone. Finally, it was found that especially an electric water heater and air conditioner had a big influence on the electrical consumption.\\

In Chapter \ref{cha:Forecasting the daily electricity consumption}, the development of baseline models and the LSTM neural networks is explained for the three consumption series with the least amount of missing data. The baseline model performing best on all error metrics except MAPE was the ``mean forecast'' and the baseline model performing best on the MAPE metric was the ``MAPE forecast''. Both baseline models predict the trend line and don't accurately forecast peaks. Two of the three LSTM models developed were stateless with Model 2 having an additional flattening layer in comparison to Model 1. Model 3 is a stateful model, which makes use of seeding before it makes predictions. A parameter search is conducted where tuning the learning rate often contributed most to model performance.\\

In Chapter \ref{cha:Model evaluation}, the models that were retrieved were evaluated on a test set consisting out of 30 days. The three LSTM models outperformed both the baseline models for Serie 2 and 3 based on the MAE metric. The flattening layer didn't lead to much improvement because the second model always performed slightly worse than Model 1 and 3 based on the MAE. All three LSTM models are able to predict peaks with a higher precision than the baseline models. The predictions of the LSTM neural networks are often an overestimation of the reference signal. Especially for Serie 1, the shape of the reference and prediction signals are similar, but there is an large offset between the two signals at the start. The offset could possibly be reduced with another choice of error metric during training. However, it can be a serie dependent effect because the large offset at the start only occurred for Serie 1. Because of the overestimation on small values, it was found that the MAPE of the LSTM neural networks was worse than the baseline models. It can be argued that in practice it is better to overestimate, than underestimate. Also, the influence of adding regularization could be seen which leads to more smooth signals. The ``mean forecast'' baseline model was not able to predict the peaks in the reference signal as good as the LSTM models, but it has a lower offset error in Serie 1. The ``MAPE forecast'' is focussed on predicting all the small consumption values correct to minimize the MAPE, but ignores all the peaks of the reference signal.

The practical usefulness of the LSTM neural networks is the ability to better monitor the demand side of the low voltage grid. Scheduling of maintenance work on the grid can be optimized, which needs detailed forecasting of the consumption series of only a small amount of households. Expenses can be saved by implementing customized updates to the network. Individual household predictions can be used as part of a congestion prediction, which means predicting when the low voltage grid can't handle the demand anymore. Especially, predicting the peaks in the household electrical consumption is in that case important. If an accurate congestion prediction can be made, reliability of the network can be increased and the risk for blackouts and brownouts is decreased. Because the thesis is developed together with the STADIUS Research Group at the KU Leuven, it is in line with their research and it can be used to build further on.\\

In Chapter \ref{cha:Forecasting the daily electricity consumption} a parameter search is conducted where parameter values are manually chosen. In order to use an automated approach which can lead to better results, a genetic algorithm can be developed with as objective function the performance of the candidate model on a validation set. Also, a more extensive, manual parameter search, where the synergy between different parameters is not neglected as was the case in Section \ref{s:Parameter search}, can be tried.\\
The LSTM neural network is a black box model and it is not immediately clear which inputs contribute most to good results. Research can be conducted to which inputs contribute most to the results of the model. A final suggestion for future work is the investigation of the effect of pooling neighbouring households as was described in \cite{Shi2018}. When spatial information about the households is absent, clustering could be done based on the exogenous variables e.g. temperature together with the assumption that two households with similar exogenous series lay close to each other. \\

The final suggestion of future work is the use of a CNN-LSTM as discussed in Chapter \ref{cha:State of the art short-term residential load forecasting techniques}, where the CNN is used  to remove noise from the inputs and extract features that afterwards are feeded to a LSTM. 




%%% Local Variables: 
%%% mode: latex
%%% TeX-master: "thesis"
%%% End: 
