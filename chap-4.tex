\chapter{Evaluation results}
\label{cha:Evaluating results}


\section{Model selection}
- the model is run multiple times and the last ten days of November are used to evaluate the performance of the model run. Also, $ 10 \% $ of the data is used for early stopping with a patience value of two. This means that when the validation set increases during two epochs, the training is stopped. The maximum amount of epochs during training is set on $ 150 $.


\subsubsection{Evaluation}




Run the best model each time 20 times and report the difference of  performance on the validationset by making use of a boxplot. 


Say that here use the models that obtained in the previous chapter. Make MSE plot for the different NN and the baseline models for each serie and make a comparison of the MAPE with the other baseline model. (by making use of barplots --> normalized with the worst performer) Performance on the test set. 

Make a plot of the day forecast for each NN model --> four graphs bundled. 
- for comparing performance in the same time serie --> mae is used. To compare performance between different time series --> MAPE is used. 
- for stateless random 10 percent is used as validation set. Remember that should shuffle the training set beforehand otherwise just the last 9 percent of the data is used. this is around 9 days
- for stateful --> the last 30 days of November will be used --> no shuffling is allowed.

- make a graph which shows all the different days that have to be forecasted on the x-axis and the MAE error on the y-axis for the different models. (line graph)

- give an indication how long the models trained.

- when there are a lot of missing days in the test data: serie 1 (0 days) and the other two (8days) -->it is naturally that the model performs worse, the previous day it bases its forecast on is just an estimate. For the day that is forecasted are always days where the true reference signal is available. 


\section{Conclusion}
\lipsum[86-88]

%%% Local Variables: 
%%% mode: latex
%%% TeX-master: "thesis"
%%% End: 
